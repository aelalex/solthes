\documentclass{article}

% Language setting
% Replace `english' with e.g. `spanish' to change the document language
\usepackage[english]{babel}

% Set page size and margins
% Replace `letterpaper' with `a4paper' for UK/EU standard size
\usepackage[letterpaper,top=2cm,bottom=2cm,left=3cm,right=3cm,marginparwidth=1.75cm]{geometry}

% Useful packages
\usepackage{amsmath}
\usepackage{graphicx}
\usepackage[colorlinks=true, allcolors=blue]{hyperref}

\title{Validation of SOLTHES - Solar Thermal Systems Educational Software with TRNSYS}
\author{Alexandros Tsimpoukis}
\date{ }
\begin{document}
\maketitle

% \begin{abstract}
% Your abstract.
% \end{abstract}

\section{Introduction}
SOLTHES is an in-house MATLAB code designed to simulate typical solar thermal systems and provide insights into their performance and energy efficiency. Here, the software is compared with the TRNSYS software \cite{Klein2020}, a powerful software for simulation of solar thermal systems and the F-chart method \cite{Beckman2021}, a well-established tool for estimating solar thermal systems. The tested example is about the New York city and the comparison is made for the solar fraction $f$ and the average collector efficiency $\bar{n}$ with the TRNSYS software and the solar fraction $f$ with the F-chart method.

The input data for the example are:
\begin{itemize}
\item
    Collector: ${{\left( \tau \alpha  \right)}_{eff}}=0.822,$ $U=5.33$ $\text{W/(m}^2 \text{K})$,${{F}_{R}}=0.9$, slope $40^0$, $\gamma =0$, diffuse reflectance ${{\rho }_{r}}=0.2$. The mass flow rate is given as function of the collector area: ${{m}_{c}}={A}/{60}$ ${\text{kg}}/{\text{s}}$.
\item   
    Storage: ${{U}_{st}}=0.35$ $\text{W/(m}^2 \text{K})$, length to diameter ratio ${{{L}_{st}}}/{{{R}_{st}}}=3$, initial temperature ${{T}_{st,initial}}=300\ \text{K}$, maximum temperature ${{T}_{st,\max }}=373\ \text{K}$, the other initial temperatures of the system are considered as equal to the initial storage temperature ${{T}_{st,initial}}={{T}_{f,in}}={{T}_{f,out}}={{T}_{l,in}}={{T}_{l,out}}=300\ \text{K}$.
\item 
    Space heating load: degree-hour method $\dot{L}={{U}_{house}}{{A}_{house}}\left( {{T}_{b}}-{{T}_{a}} \right)$ $\text{Whr}$, building loss coefficient ${{U}_{house}}=0.55$ $\text{W/(m}^2 \text{K})$, area of the building envelope ${{A}_{house}}=320$ $\text{m}^2$, balance point temperature ${{T}_{b}}=18.3$ $^{\text{0}}\text{C}$, minimum temperature to cover load ${{T}_{l,min}}=300\ \text{K}$ and mass flow rate of the load: ${{m}_{l}}=0.0166$ ${\text{kg}}/{\text{s}}$.
\item 
    Domestic hot water: The consumption is 100 lt per person per day at a temperature of $45$ $^{\text{0}}\text{C}$. The persons considered in this example are 8. The hourly consumption is given as a percentage of the total consumption: 40\% at 7-9am, 20\% at 12π.μ-2pm, and 40\% at 6-8pm. The supply temperature is $10$ $^{\text{0}}\text{C}$.
\item 
    The data for the F-chart method are taken from Appendix D of \cite{Beckman2021}. The solar radiation on a sloped surface, with a slope of $40^0$, is calculated and presented in Table \ref{tab:Solar_radiation}.
 \end{itemize}

\begin{table}
\caption{\label{tab:Solar_radiation} Solar radiation on a sloped surface}
\centering
\begin{tabular}{|l|l|l|l|l|l|l|l|l|l|l|l|l|}
\hline
Month & Jan & Feb & Mar & Apr & May & Jun & Jul & Aug & Sep & Oct & Nov & Dec \\\hline
MJ/day &  8.74 & 11.03 & 13.67 & 15.50 & 16.51 & 16.32 & 16.45 & 15.93 & 15.09 & 13.30 & 8.78 & 7.05 \\
\hline
\end{tabular}
\end{table}

\section{Comparison with TRNSYS}

In this Section, a comparison with the TRNSYS software is made. More specifically, the Combisys version of the TRNSYS software included in \cite{Beckman2021} is used. In order for the simulations to be equivalent, some problem parameters are considered as zero in TRNSYS since the SOLTHES software cannot calculate them. The first one is incidence angle modifier and the second one is the Second-Order Loss Coefficient. The effectiveness of the heat exchanger between the collector is considered as one and the storage tank is modelled as a one node tank.

In Table \ref{tab:Solar_fraction_heating_load}, the solar fraction $f$ and the average collector efficiency $\bar{n}$ are presented for the in-house code and the TRNSYS software in terms of several simulation periods, ranging from  a single day to one year. Only the space heating load is considered here. It is seen that the deviation between the two software is large for a single day it reaches up to 13\% for the solar fraction and 4\% for the collector efficiency. As the simulation period is increased, the deviations are decreased and the results are almost identical. The initial values of the quantities have a significant effect on the results for a small simulation period. However, as the simulation period is increased then the effect of the initial values is diminished and the comparison between the two software is very good. In addition, the small difference in the average collector efficiency is due to deviations between the calculated solar radiation on a sloped surface values. Even though, the model for calculating the solar radiation on a sloped surface is the same for both software (Isotropic sky model), there is still a small difference 1-2\% in the calculated values.   



\begin{table}
\caption{\label{tab:Solar_fraction_heating_load} Solar fraction $f$ and average collector efficiency $\bar{n}$ for space heating load}
\centering
\begin{tabular}{|l|l|l|l|l|}
\hline
Simulation time & TRNSYS $f$ & TRNSYS $\bar{n}$  & Solsys $f$	& Solsys $\bar{n}$ \\\hline
1 day	& 0.61 & 0.33 & 0.48 & 0.37 \\\hline
1 week & 0.26 &	0.29 & 0.25 & 0.33 \\\hline
1 month & 0.18 &	0.26 &	0.18 & 0.29 \\\hline
2 months & 0.21 & 0.27 &	0.21 & 0.29 \\\hline
1 year & 0.35 &	0.15 &	0.35 &	0.17 \\\hline
\end{tabular}
\end{table}

In Table \ref{tab:Solar_fraction_combined_load}, the solar fraction $f$ and the collector efficiency $\bar{n}$ are presented for the in-house code and the TRNSYS software in terms of several simulation periods, ranging from single day to one year. The load here is the sum of the space heating and the domestic hot water load. It is seen that there are deviations between the two software, which are reduced as the simulation period is increased. For the smaller simulation periods, except for the single day, the solar fraction is reduced due to the increased space heating load. However, the solar fraction $f$ for the 1 year simulation is increased since the solar thermal system covers the domestic hot water load. The space heating load at the summer is zero leading to an increased annual solar fraction.

\begin{table}
\caption{\label{tab:Solar_fraction_combined_load} Solar fraction $f$ and average collector efficiency $\bar{n}$ for a combined load, space heating and domestic hot water}
\centering
\begin{tabular}{|l|l|l|l|l|}
\hline
Simulation time & TRNSYS $f$ & TRNSYS $\bar{n}$  & SOLTHES $f$	& SOLTHES $\bar{n}$ \\\hline
1 day	& 0.49 & 0.36 & 0.34 & 0.4 \\\hline
1 week & 0.21 &	0.32 & 0.2 & 0.36 \\\hline
1 month & 0.14 &	0.29 &	0.14 & 0.3 \\\hline
2 months & 0.17 & 0.31 &	0.16 & 0.31 \\\hline
1 year & 0.39 &	0.31 &	0.39 &	0.31 \\\hline
\end{tabular}
\end{table}

\section{Comparison with F-chart}
In this Section a comparison of the SOLTHES software with the F-chart method is done. At first, the monthly solar fraction $f$ is presented in Table \ref{tab:Fraction_heat} for SOLTHES software and the F-chart method. It is seen that the differences between the in-house software and the f-chart method are increased in the spring and decreased in the winter and the autumn. One of the possible causes for the deviation is that the monthly data used in the f-chart method can not capture the daily fluctuations of the weather that a time-dependent simulation software can.    


\begin{table}
\centering
\caption{\label{tab:Fraction_heat} Solar fraction $f$ for space heating load}
\begin{tabular}{|l|l|l|l|l|l|l|l|l|l|l|l|l|}
\hline
Month & Jan & Feb & Mar & Apr & May & Jun & Jul & Aug & Sep & Oct & Nov & Dec \\\hline
F-chart & 0.20 & 0.31 & 0.52 & 0.82 & 1.00 & 1.00 & 1.00 & 1.00 & 1.00 & 0.87 & 0.36 & 0.14  \\\hline
SOLTHES & 0.17 & 0.29 & 0.38 & 0.68 & 0.92 &	1.00 & 1.00 & 1.00 & 0.97 & 0.80 & 0.30 & 0.18  \\\hline
\end{tabular}
\end{table}

In Table \ref{tab:Fraction_dhw}, the monthly solar fraction $f$ is presented for SOLTHES software and the F-chart method. Here, another configuration for the SOLTHES software (b) is considered, where there is a change in the control function from ${{T}_{st}}<{{T}_{l\min }}$ to  ${{T}_{st}}<{{T}_{\sup }}$ due to the domestic hot water load. It is seen that there is a larger difference between the months March to September than the other ones. On the contrary, the other configuration, SOLTHES (b), has a better behavior from March to September. It seems that the load circuit closes at lower temperatures and this control scheme helps the system operate better in the spring and summer months. Therefore, depending on the simulation period some tweaks concerning the control scheme may be required. If there is a combined load of space heating and domestic hot water, the initial configuration is needed as it has been seen in Table \ref{tab:Solar_fraction_combined_load}.

\begin{table}
\caption{\label{tab:Fraction_dhw} Solar fraction $f$ for domestic hot water load}
\centering
\begin{tabular}{|l|l|l|l|l|l|l|l|l|l|l|l|l|}
\hline
Month & Jan & Feb & Mar & Apr & May & Jun & Jul & Aug & Sep & Oct & Nov & Dec \\\hline
F-chart &  0.41 & 0.57 & 0.73 & 0.86 & 0.95 & 1.00 & 1.00	& 1.00 & 0.93 & 0.80 & 0.48 & 0.29  \\\hline
SOLTHES & 0.35 &	0.51 & 0.59 & 0.76 & 0.80 &	0.94 & 0.94 & 0.93 & 0.85 &	0.73 &	0.39 &	0.35  \\\hline
SOLTHES (b) & 0.50 & 0.67 &	0.74 &	0.89 &	0.92 &	0.99 &	0.99 &	0.99 &	0.95 &	0.52 &	0.59 &	0.52 \\\hline
\end{tabular}
\end{table}

\section{Concluding remarks}
SOLTHES is a software tool developed for modeling solar thermal systems. It is designed to accurately simulate the behavior and performance of solar thermal systems based on the input parameters and system configurations provided by the user. Overall, it is seen that the comparison between the SOLTHES and the TRNSYS software as well as the F-chart method is very good for large simulation periods and therefore it can correctly model a typical solar thermal system by taking into account the weather data of each site. 
\bibliographystyle{unsrt}
\bibliography{references}

\end{document}